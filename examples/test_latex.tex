\documentclass{article}
\usepackage{fancyvrb}
\title{Test Program}
\author{Jason R. Fruit}

\begin{document}

\maketitle
\tableofcontents

\section{Introduction}

This test program prints the word ``hello'', followed by the name of
the operating system as understood by Python.  It is implemented in
Python and uses the \texttt{os} module.  It builds the message string
in two different ways, and writes separate versions of the program to
two different files.

\section{Implementation}

\subsection{Output files}

This document contains the makings of two files; the first,
\texttt{test.py}, uses simple string concatenation to build its output
message:

\label{pyweb1}
    \begin{flushleft}
    \textit{Code example test.py (1)}
    \begin{Verbatim}[commandchars=\\\{\},codes={\catcode`$=3\catcode`^=7},frame=single]

$\triangleright$ Code Example Import the os module (3)
$\triangleright$ Code Example Get the OS description (4)
$\triangleright$ Code Example Construct the message with Concatenation (5)
$\triangleright$ Code Example Print the message (7)

    \end{Verbatim}
    
    \end{flushleft}


The second uses string substitution:

\label{pyweb2}
    \begin{flushleft}
    \textit{Code example test2.py (2)}
    \begin{Verbatim}[commandchars=\\\{\},codes={\catcode`$=3\catcode`^=7},frame=single]

$\triangleright$ Code Example Import the os module (3)
$\triangleright$ Code Example Get the OS description (4)
$\triangleright$ Code Example Construct the message with Substitution (6)
$\triangleright$ Code Example Print the message (7)

    \end{Verbatim}
    
    \end{flushleft}


\subsection{Retrieving the OS description}

First we must import the os module so we can learn about the OS:

\label{pyweb3}
    \begin{flushleft}
    \textit{Code example Import the os module (3)}
    \begin{Verbatim}[commandchars=\\\{\},codes={\catcode`$=3\catcode`^=7},frame=single]

import os

    \end{Verbatim}
    
    \footnotesize
    Used by:
    \begin{list}{}{}
    
    \item Code example test.py (1) (Sect. \ref{pyweb1}, p. \pageref{pyweb1})
; 
    \item Code example test2.py (2) (Sect. \ref{pyweb2}, p. \pageref{pyweb2})

    \end{list}
    \normalsize
    
    \end{flushleft}


That having been done, we can retrieve Python's name for the OS type:

\label{pyweb4}
    \begin{flushleft}
    \textit{Code example Get the OS description (4)}
    \begin{Verbatim}[commandchars=\\\{\},codes={\catcode`$=3\catcode`^=7},frame=single]

os_name = os.name

    \end{Verbatim}
    
    \footnotesize
    Used by:
    \begin{list}{}{}
    
    \item Code example test.py (1) (Sect. \ref{pyweb1}, p. \pageref{pyweb1})
; 
    \item Code example test2.py (2) (Sect. \ref{pyweb2}, p. \pageref{pyweb2})

    \end{list}
    \normalsize
    
    \end{flushleft}


\subsection{Building the message}

Now, we're ready for the meat of the application: concatenating two strings:

\label{pyweb5}
    \begin{flushleft}
    \textit{Code example Construct the message with Concatenation (5)}
    \begin{Verbatim}[commandchars=\\\{\},codes={\catcode`$=3\catcode`^=7},frame=single]

msg = "Hello, " + os_name + "!"

    \end{Verbatim}
    
    \footnotesize
    Used by:
    \begin{list}{}{}
    
    \item Code example test.py (1) (Sect. \ref{pyweb1}, p. \pageref{pyweb1})

    \end{list}
    \normalsize
    
    \end{flushleft}


But wait!  Is there a better way?  Using string substitution might be
better:

\label{pyweb6}
    \begin{flushleft}
    \textit{Code example Construct the message with Substitution (6)}
    \begin{Verbatim}[commandchars=\\\{\},codes={\catcode`$=3\catcode`^=7},frame=single]

msg = "Hello, %s!" % os_name

    \end{Verbatim}
    
    \footnotesize
    Used by:
    \begin{list}{}{}
    
    \item Code example test2.py (2) (Sect. \ref{pyweb2}, p. \pageref{pyweb2})

    \end{list}
    \normalsize
    
    \end{flushleft}


We'll use the first of these methods in \texttt{test.py}, and the
other in \texttt{test2.py}.

\subsection{Printing the message}

Finally, we print the message out for the user to see.  Hopefully, a
cheery greeting will make them happy to know what operating system
they have:

\label{pyweb7}
    \begin{flushleft}
    \textit{Code example Print the message (7)}
    \begin{Verbatim}[commandchars=\\\{\},codes={\catcode`$=3\catcode`^=7},frame=single]

print msg

    \end{Verbatim}
    
    \footnotesize
    Used by:
    \begin{list}{}{}
    
    \item Code example test.py (1) (Sect. \ref{pyweb1}, p. \pageref{pyweb1})
; 
    \item Code example test2.py (2) (Sect. \ref{pyweb2}, p. \pageref{pyweb2})

    \end{list}
    \normalsize
    
    \end{flushleft}


\end{document}
