% generated by Docutils <http://docutils.sourceforge.net/>
\documentclass[a4paper,english]{article}
\usepackage{fixltx2e} % LaTeX patches, \textsubscript
\usepackage{cmap} % fix search and cut-and-paste in PDF
\usepackage[T1]{fontenc}
\usepackage[utf8]{inputenc}
\usepackage{ifthen}
\usepackage{babel}
\usepackage{textcomp} % text symbol macros

%%% Custom LaTeX preamble
% PDF Standard Fonts
\usepackage{mathptmx} % Times
\usepackage[scaled=.90]{helvet}
\usepackage{courier}

%%% User specified packages and stylesheets

%%% Fallback definitions for Docutils-specific commands

% rubric (informal heading)
\providecommand*{\DUrubric}[2][class-arg]{%
  \subsubsection*{\centering\textit{\textmd{#2}}}}

% hyperlinks:
\ifthenelse{\isundefined{\hypersetup}}{
  \usepackage[unicode,colorlinks=true,linkcolor=blue,urlcolor=blue]{hyperref}
  \urlstyle{same} % normal text font (alternatives: tt, rm, sf)
}{}
\hypersetup{
  pdftitle={Test Program},
}

%%% Body
\begin{document}

% Document title
\title{Test Program%
  \phantomsection%
  \label{test-program}%
  \\ % subtitle%
  \large{Jason R. Fruit}%
  \label{jason-r-fruit}}
\author{}
\date{}
\maketitle

% This data file has been placed in the public domain.

% Derived from the Unicode character mappings available from
% <http://www.w3.org/2003/entities/xml/>.
% Processed by unicode2rstsubs.py, part of Docutils:
% <http://docutils.sourceforge.net>.

\phantomsection\label{contents}
\pdfbookmark[1]{Contents}{contents}
\tableofcontents



%___________________________________________________________________________

\section*{Introduction%
  \phantomsection%
  \addcontentsline{toc}{section}{Introduction}%
  \label{introduction}%
}

This test program prints the word ``hello'', followed by the name of
the operating system as understood by Python.  It is implemented in
Python and uses the \texttt{os} module.  It builds the message string
in two different ways, and writes separate versions of the program to
two different files.


%___________________________________________________________________________

\section*{Implementation%
  \phantomsection%
  \addcontentsline{toc}{section}{Implementation}%
  \label{implementation}%
}


%___________________________________________________________________________

\subsection*{Output files%
  \phantomsection%
  \addcontentsline{toc}{subsection}{Output files}%
  \label{output-files}%
}

This document contains the makings of two files; the first,
\texttt{test.py}, uses simple string concatenation to build its output
message:

\DUrubric{test.py (1)}
%
\begin{quote}{\ttfamily \raggedright \noindent
→~Import~the~os~module~(\hyperref[id3]{3})\\
~→~Get~the~OS~description~(\hyperref[id4]{4})\\
~→~Construct~the~message~with~Concatenation~(\hyperref[id5]{5})\\
~→~Print~the~message~(\hyperref[id7]{7})
}
\end{quote}

The second uses string substitution:

\DUrubric{test2.py (2)}
%
\begin{quote}{\ttfamily \raggedright \noindent
→~Import~the~os~module~(\hyperref[id3]{3})\\
~→~Get~the~OS~description~(\hyperref[id4]{4})\\
~→~Construct~the~message~with~Substitution~(\hyperref[id6]{6})\\
~→~Print~the~message~(\hyperref[id7]{7})
}
\end{quote}


%___________________________________________________________________________

\subsection*{Retrieving the OS description%
  \phantomsection%
  \addcontentsline{toc}{subsection}{Retrieving the OS description}%
  \label{retrieving-the-os-description}%
}

First we must import the os module so we can learn about the OS:

\DUrubric{Import the os module (3)}
%
\begin{quote}{\ttfamily \raggedright \noindent
import~os
}
\end{quote}

Used by: test.py (\hyperref[id1]{1}); test2.py (\hyperref[id2]{2})

That having been done, we can retrieve Python's name for the OS type:

\DUrubric{Get the OS description (4)}
%
\begin{quote}{\ttfamily \raggedright \noindent
os\_name~=~os.name
}
\end{quote}

Used by: test.py (\hyperref[id1]{1}); test2.py (\hyperref[id2]{2})


%___________________________________________________________________________

\subsection*{Building the message%
  \phantomsection%
  \addcontentsline{toc}{subsection}{Building the message}%
  \label{building-the-message}%
}

Now, we're ready for the meat of the application: concatenating two strings:

\DUrubric{Construct the message with Concatenation (5)}
%
\begin{quote}{\ttfamily \raggedright \noindent
msg~=~"Hello,~"~+~os\_name~+~"!"
}
\end{quote}

Used by: test.py (\hyperref[id1]{1})

But wait!  Is there a better way?  Using string substitution might be
better:

\DUrubric{Construct the message with Substitution (6)}
%
\begin{quote}{\ttfamily \raggedright \noindent
msg~=~"Hello,~\%s!"~\%~os\_name
}
\end{quote}

Used by: test2.py (\hyperref[id2]{2})

We'll use the first of these methods in \texttt{test.py}, and the
other in \texttt{test2.py}.


%___________________________________________________________________________

\subsection*{Printing the message%
  \phantomsection%
  \addcontentsline{toc}{subsection}{Printing the message}%
  \label{printing-the-message}%
}

Finally, we print the message out for the user to see.  Hopefully, a
cheery greeting will make them happy to know what operating system
they have:

\DUrubric{Print the message (7)}
%
\begin{quote}{\ttfamily \raggedright \noindent
print~msg
}
\end{quote}

Used by: test.py (\hyperref[id1]{1}); test2.py (\hyperref[id2]{2})

\end{document}
